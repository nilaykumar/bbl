\chapter{Preliminaries}

\blurb{We introduce basic definitions and properties relating to $\mathscr{D}$-modules.
We then detail several functors between categories of $\mathscr{D}$-modules and related results.}

Fix $X$ a smooth complex algebraic variety. We provide a coordinate-independent definition
of certain sheaves of differential operators.

\begin{definition}
    Let $R$ be a commutative $\C$-algebra, and $M,N$ be $R$-modules.
    We define the module of $\C$-derivations $\Diff_R(M,N)$ to be the submodule
    of homomorphisms $d\in\Hom_\C(M,N)$ such that there exists an $k\in\N$ with any set
    of $k+1$ elements of $R$ satisfying
    \begin{equation}
        (\ad r_0)(\ad r_1)\cdots(\ad r_k)(d)=0,
        \label{eq:ad}
    \end{equation}
    where $(\ad r)(d)=[r,d]=rd-dr$. In this case, we say that $d$ is of order $\leqslant k$.
    The submodule of $\C$-derivations of order $\leqslant 0$, in particular, is simply
    $\Hom_R(M,N)$.
    \label{def:derivations}
\end{definition}

\begin{remark}
    The order of a $\C$-derivation provides a natural filtration of the module $\Diff_R(M,N)$.
    This yields the following inductive definition of $\Diff_R(M,N)$, equivalent to the one
    above:
    \begin{enumerate}
        \item[] $\Diff^{i}_R(M,N) = 0; \;\;\;(i<0)$
        \item[] $\Diff^k_R(M,N) = \{u\in\Hom_\C(M,N)\mid (\ad r)(u)\in \Diff^{k-1}_R(M,N)\};$
        \item[] $\Diff_R(M,N)=\bigcup_k\Diff^k_R(M,N)$.
    \end{enumerate}
\end{remark}

\begin{definition}
    Let $\mathcal{M},\mathcal{N}$ be $\mathcal{O}_X$-modules. We define the
    $\mathcal{O}_X$-module $\mathscr{D}(\mathcal{M},\mathcal{N})$ of $\C$-derivations
    from $\mathcal{M}$ to $\mathcal{N}$ by localizing the construction of Defintion \ref{def:derivations}.
    More explicitly, we define $\mathscr{D}(\mathcal{M},\mathcal{N})$ to be
    the sheafification of the subpresheaf of $\sHom_\C(\mathcal{M},\mathcal{N})$
    whose sections above an open $U\subset X$ is $\Diff_{\mathcal{O}_X(U)}(\mathcal{M}(U),\mathcal{N}(U))$.
    Note that we will often write $\mathscr{D}(\mathcal{M})\equiv\mathscr{D}(\mathcal{M,M})$ and
    $\mathscr{D}_X\equiv\mathscr{D}(\mathcal{O}_X)$.
    \label{def:D_X}
\end{definition}

\begin{lemma}
    Consider $U\subset X$ affine open. Then
    \[\mathscr{D}_X(U)\cong\Diff_{\mathcal{O}_X(U)}(\mathcal{O}_X(U),\mathcal{O}_X(U)).\]
    \label{lem:affine1}
\end{lemma}
\begin{proof}
    \sidenote{prove this!}
\end{proof}

\begin{lemma}
    Order provides the $\mathcal{O}_X$-module $\mathscr{D}(\mathcal{M})$ with
    a filtration, making it a filtered sheaf of rings.
    \label{lem:filter1}
\end{lemma}
\begin{proof}
    Fix $U\subset X$ open. Sections of $\mathscr{D}(\mathcal{M})(U)$ are $\C_X$-linear
    sheaf homomorphisms $\phi:\mathcal{M}|_U\to\mathcal{M}|_U$. Defining multiplication as composition
    of sections, we obtain a sheaf of rings. We give $\mathscr{D}(M)$ a filtration by specifying
    $F^k\mathscr{D}(\mathcal{M})$ to be the subsheaf consisting of morphisms which, over every affine
    open $V$, lie in $\Diff^k_{\mathcal{O}_X(V)}(\mathcal{M}(V),\mathcal{M}(V))$. It is easy to see
    that restriction maps preserve the filtration and that $\cup_kF^k\mathscr{D}(\mathcal{M})=\mathscr{D}(\mathcal{M})$.
    It remains to check that
    $F^i\mathscr{D}(\mathcal{M})\cdot F^j\mathscr{D}(\mathcal{M})\subset F^{i+j}\mathscr{D}(\mathcal{M})$.
    This follows\sidenote{explain this better} from the filtration on $\Diff_R(M,M)\equiv\Diff_R(M)$ (for $M$ an $R$-module);
    if $\alpha\in\Diff^i_R(M),\beta\in\Diff^j_R(M),r\in R$,
    \[ [\alpha\beta,r] = \alpha\beta r-r\alpha\beta = \alpha[\beta,r] +[\alpha,r]\beta.  \]
    Now $[\alpha,r]$ is of order $\leqslant (i-1)$ and $[\beta,r]$ is of order $\leqslant(j-1)$, so induction
    yields the case of degree $i$ times an $R$-linear endomorphism $f$ (degree zero), but
    \[ [\alpha f, r]=\alpha fr-r\alpha f=\alpha rf-r\alpha f=[\alpha,r]f, \]
    so induction yields the case of the product two $R$-linear endomorphisms, which is another
    $R$-linear endomorphism, hence degree $0$.
\end{proof}

\begin{remark}
    The sheaf of rings $\mathscr{D}(\mathcal{M})$ is not, in general, commutative or even
    almost-commutative (that is to say, $\gr_F \mathscr{D}(\mathcal{M})$ is not commutative).
    This is due to the fact that the endomorphism ring $\Diff_R^0(M)=\End_R(M)$ need not be
    commutative.\sidenote{Perhaps it is worth mentioning the alternate definition of $\Diff$ here? We
    should learn more about the two definitions, although probably it is no big deal since
    they yield the same thing for $\mathscr{D}_X$.}
\end{remark}

\begin{proposition}
    If $\mathcal{M},\mathcal{N}$ are locally free, we obtain an isomorphism
    \[\mathscr{D}(\mathcal{M},\mathcal{N})=\mathcal{N}\otimes_{\mathcal{O}_X}\mathscr{D}_X\otimes_{\mathcal{O}_X}\mathcal{M}^*.\]
\end{proposition}
\begin{proof}
    \sidenote{prove this!}
\end{proof}

Recall that we have defined $\mathscr{D}_X\equiv\mathscr{D}(\mathcal{O}_X,\mathcal{O}_X)$.
An often-used definition equivalent to the above is as follows.\sidenote{why is this equivalent?} 

\begin{definition}
    Define $\mathscr{D}_X$ to be the sheaf associated to the subpresheaf of $\sHom_{\C_X}(\mathcal{O}_X,\mathcal{O}_X)$
    generated as a $\C_X$-algebra by (multiplication by) $\mathcal{O}_X$ and (derivation by)
    $\Theta_X$. Explicitly, for any open $U\subset X$, the sections of $\mathscr{D}_X(U)$ are
    the endomorphisms which are locally generated -- i.e. over an open cover $U_i$ of $U$ depending
    on the endomorphism -- by 
    $\{\tilde f,\tilde \partial\mid f\in\mathcal{O}_X(U_i),\partial\in\Theta_X(U_i)\}$
    subject to the relations
    \begin{enumerate}[(i)]
        \item $\tilde f_1+\tilde f_2=\widetilde{f_1+f_2}, \;\;\; \tilde f_1\tilde f_2=\widetilde{f_1f_2};$
        \item $\tilde \partial_1 +\tilde \partial_2=\widetilde{\partial_1+\partial_2}, \;\; [\tilde\partial_1,\tilde\partial_2]=\widetilde{[\partial_1,\partial_2]};$
        \item $\tilde f\tilde \partial=\widetilde{f\partial}, \;\;\; [\tilde\partial,\tilde f]=\widetilde{\partial(f)}$.
    \end{enumerate}
    \label{def:D_X2}
\end{definition}

Before we prove that our two definitions are equivalent, we prove the following
lemma that is useful for local computations.
\begin{lemma}
    Consider $U\subset X$ open affine. Then $\mathscr{D}_X(U)$ is precisely
    the algebra generated by $\mathcal{O}_X(U)$ and $\Theta_X(U)$ as above.
    \label{lem:affine2}
\end{lemma}
\begin{proof}
    \sidenote{Prove this!}
\end{proof}

This definition is especially useful due to the following coordinate description.

\begin{lemma}
    For each $p\in X$ there exists an affine open $U\ni p$ and sections
    $x_i\in\mathcal{O}_X(V),\partial_i\in\Theta_X(V)$ (for $1\leq i\leq n=\dim X$) satisfying
    \begin{enumerate}[(i)]
        \item $[\partial_i,\partial_j]=0$;
        \item $\partial_i(x_j)=\delta_{ij}$;
        \item $\Theta_X(V)=\bigoplus_{i=1}^n\mathcal{O}_X(V)\partial_i$.
    \end{enumerate}
\end{lemma}
\begin{proof}
    Since $X$ is smooth, the stalk $\mathcal{O}_{X,p}$ is a regular local ring. Then the maximal
    ideal $\fr m_p\subset \mathcal{O}_{X,p}$ is generated by $n$ elements, call them
    $x_1,\ldots,x_n\in\fr m_p$. Then $\Omega^1_{X,p}$ is a free $O_{X,p}$-module generated
    by $dx_1,\ldots, dx_n$.\sidenote{explain this} Thus there exists an affine open neighboorhood
    $V$ of $p$ such that $\Omega_X(V)$ is a free $\mathcal{O}_X(V)$-module with basis
    $dx_1,\ldots,dx_n$.\sidenote{why?} Fixing the dual basis $\partial_1,\ldots,\partial_n\in\Theta_X(V)$
    satisfying $dx_i(\partial_j)=\delta_{ij}$, we find that $\partial_j(x_i)=\delta_{ij}$.\sidenote{finish}
\end{proof}

\begin{lemma}
    Under this definition, $\mathscr{D}_X$ becomes a filtered sheaf of rings
    via the power of derivations.
    \label{lem:filter2}
\end{lemma}
\begin{proof}
    \sidenote{Prove this!}
\end{proof}

Let us now show that the two definitions of $\mathscr{D}_X$ are equivalent.

\begin{lemma}
    The sheaf of rings generated by $\mathcal{O}_X$ and $\Theta_X$ in Definition
    \ref{def:D_X2} is isomorphic to the sheaf of rings $\mathscr{D}_X=\mathscr{D}(\mathcal{O}_X)$
    of Definition \ref{def:D_X}.
\end{lemma}
\begin{proof}
    It suffices to show that the sheaves have isomorphic sections on the basis
    of affine opens. By Lemmas \ref{lem:affine1} and \ref{lem:affine2}, it suffices to
    find an isomorphism
    \[\Diff_{\mathcal{O}_X(U)}(\mathcal{O}_X(U),\mathcal{O}_X(U)) \cong \{\tilde f,\tilde \partial\mid f\in\mathcal{O}_X(U),\partial\in\Theta_X(U)\}/\sim\]
    for $U\subset X$ affine open. For notational convenience, denote the
    algebra on the left by $A$ and the algebra on the right by $B$. There is a
    natural inclusion $B\hookrightarrow A$, as any product of derivations of
    $\mathcal{O}_X(U)$ must satisfy (\ref{eq:ad}).
    Both $A$ and $B$ are filtered by Lemmas \ref{lem:filter1} and \ref{lem:filter2}
    -- we prove that the inclusion $B\hookrightarrow A$ is an isomorphism by
    induction on the filtrations. 
\end{proof}

