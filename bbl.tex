% This work is licensed under the Creative Commons Attribution-NonCommercial-ShareAlike 3.0 Unported License.
% To view a copy of this license, visit http://creativecommons.org/licenses/by-nc-sa/3.0/ or send a letter to Creative Commons, PO Box 1866, Mountain View, CA 94042, USA.

\documentclass{article}

\usepackage{amsmath, amsthm, amssymb}
\usepackage[mathscr]{euscript}
\usepackage[colorlinks]{hyperref}
\usepackage{mathrsfs}
\usepackage{enumerate}
\usepackage{todonotes}
\usepackage{tikz-cd}

%\usepackage[style=authoryear]{biblatex}
%\addbibresource{references.bib}

\newcommand{\catname}[1]{\normalfont\textbf{#1}}
\newcommand{\C}{\mathbb{C}}
\newcommand{\N}{\mathbb{N}}
\newcommand{\fr}{\mathfrak}

\theoremstyle{plain}
\newtheorem{theorem}{Theorem}[section]
\newtheorem{lemma}[theorem]{Lemma}
\newtheorem{proposition}[theorem]{Proposition}
\newtheorem{corollary}{Corollary}

\theoremstyle{definition}
\newtheorem{definition}[theorem]{Definition}
\newtheorem{example}[theorem]{Example}
\newtheorem{exercise}{Exercise}[section]

\theoremstyle{remark}
\newtheorem*{remark}{Remark}

\DeclareMathOperator{\Gr}{Gr}
\DeclareMathOperator{\Fl}{Fl}
\DeclareMathOperator{\PP}{\mathbb{P}}
\DeclareMathOperator{\Der}{Der}
\DeclareMathOperator{\End}{End}
\DeclareMathOperator{\Hom}{Hom}
\DeclareMathOperator{\sHom}{\mathscr{H}\!om}
\DeclareMathOperator{\id}{id}
\DeclareMathOperator{\Diff}{Diff}
\DeclareMathOperator{\gr}{gr}
\DeclareMathOperator{\im}{im}
\DeclareMathOperator{\ad}{ad}
\DeclareMathOperator{\rk}{rk}

\DeclareMathOperator{\SL}{SL}
\DeclareMathOperator{\GL}{GL}

\begin{document}

\title{Beilinson-Bernstein Localization [DRAFT]}
\author{Matei Ionita and Nilay Kumar}
\date{Last updated: \today}

\maketitle

\subsection{Preliminaries}

Fix $X$ a smooth complex algebraic variety. We provide a coordinate-independent definition
of certain sheaves of differential operators.

\begin{definition}
    Let $R$ be a commutative $\C$-algebra, and $M,N$ be $R$-modules.
    We define the module of $\C$-derivations $\Diff_R(M,N)$ to be the submodule
    of homomorphisms $d\in\Hom_\C(M,N)$ such that there exists an $k\in\N$ with any set
    of $k+1$ elements of $R$ satisfying
    \[(\ad r_0)(\ad r_1)\cdots(\ad r_k)(d)=0,\]
    where $(\ad r)(d)=[r,d]=rd-dr$. In this case, we say that $d$ is of order $\leqslant k$.
    The submodule of $\C$-derivations of order $\leqslant 0$, in particular, is simply
    $\Hom_R(M,N)$.
    \label{def:derivations}
\end{definition}

\begin{remark}
    The order of a $\C$-derivation provides a natural filtration of the module $\Diff_R(M,N)$.
    This yields the following inductive definition of $\Diff_R(M,N)$, equivalent to the one
    above:
    \begin{enumerate}
        \item[] $\Diff^{i}_R(M,N) = 0; \;\;\;(i<0)$
        \item[] $\Diff^k_R(M,N) = \{u\in\Hom_\C(M,N)\mid (\ad r)(u)\in \Diff^{k-1}_R(M,N), r\in R\};$
        \item[] $\Diff_R(M,N)=\bigcup_k\Diff^k_R(M,N)$.
    \end{enumerate}
\end{remark}

\begin{definition}
    Let $\mathcal{M},\mathcal{N}$ be $\mathcal{O}_X$-modules. We define the $\mathcal{O}_X$-module $\mathcal{D}(\mathcal{M},\mathcal{N})$ of $\C$-derivations
    from $\mathcal{M}$ to $\mathcal{N}$ by localizing the construction of Defintion \ref{def:derivations}.
    More explicitly, we define $\mathcal{D}(\mathcal{M},\mathcal{N})$ to be the subsheaf of $\sHom_\C(\mathcal{M},\mathcal{N})$
    whose sections on an open set $U\subset X$ are morphisms whose evaluations at $U\cap V$, for every
    open affine $V$, are $\Diff_{\mathcal{O}_X(U\cap V)}(\mathcal{M}(U\cap V),\mathcal{N}(U\cap V))$.
    We will often write $\mathcal{D}(\mathcal{M})\equiv\mathcal{D}(\mathcal{M,M})$ and $\mathcal{D}_X\equiv\mathcal{D}(\mathcal{O}_X)$.
\end{definition}
\begin{proof}
    Let us check that $\mathcal{D}(\mathcal{M},\mathcal{N})$ is indeed a sheaf. It is clearly
    a presheaf when equipped with the restriction maps inherited from $\sHom_\C(\mathcal{M},\mathcal{N})$.
    Moreover, it is separated by virtue of being a subpresheaf of $\sHom_\C(\mathcal{M},\mathcal{N})$.
    Hence it suffices to check that sections glue. Let $U\subset X$ be open, $\{U_i\}_{i\in I}$ be
    an open cover for $U$, and $s_i\in\mathcal{D}(\mathcal{M},\mathcal{N})(U_i)$ agreeing on
    overlaps. The sheaf condition
    on $\sHom_\C(\mathcal{M},\mathcal{N})$ yields a section $s\in\sHom_\C(\mathcal{M},\mathcal{N})(U)$
    such that $s|_{U_i}=s_i$. It remains to show that $s\in\mathcal{D}(\mathcal{M},\mathcal{N)}(U)$, that is
    to say, for any open affine $V\subset X$, $s|_{U\cap V}\in\Diff_{\mathcal{O}_X(U\cap V)}(\mathcal{M}(U\cap V),\mathcal{N}(U\cap V))$.
    Writing $V_i=V\cap U_i$ we obtain $\C$-derivations $d_i=s_i|_{V_i}(V_i)$ of order $\leqslant k_i$. Taking
    $k=\max_{i\in I} k_i$ to be the maximum order of these derivations\footnote{this is bullshit}, we claim that
    \begin{equation*}
        (\ad f_0)(\ad f_1)\cdots(\ad f_k)(s|_{U\cap V})=0\tag{$\star$}
    \end{equation*}
    for any $k+1$ regular functions $f_i\in\mathcal{O}_X(V)$. To see this, fix $m\in\mathcal{M}(V)$
    with restrictions $m_i\in\mathcal{M}(V_i)$. Then the restriction of the left-hand side of ($\star$)
    to $V_i$ sends $m_i$ to $0\in\mathcal{N}(V_i)$. The commutativity of sheaf homomorphisms,
    \begin{equation*}
        \begin{tikzcd}
            \mathcal{M}(V)\ar{r}{s|_{U\cap V}}\ar{d}{\rho^\mathcal{M}_{V_iV}} & \mathcal{N}(V)\ar{d}{\rho^\mathcal{N}_{V_iV}}\\
            \mathcal{M}(V_i)\ar{r}{d_i} & \mathcal{N}(V_i)
        \end{tikzcd}
    \end{equation*}
    together with the sheaf condition on $\mathcal{N}$ forces the left-hand side of $(\star)$
    to send $m$ to $0\in\mathcal{N}(V).$ Since $m$ is arbitrary, we obtain the equality ($\star$),
    whence $s|_{U\cap V}\in\Diff_{\mathcal{O}_X(U\cap V)}(\mathcal{M}(U\cap V),\mathcal{N}(U\cap V))$
    and $s\in\mathcal{D}(\mathcal{M},\mathcal{N})(U)$.
\end{proof}

\begin{remark}
    One might instead define the sheaf $\mathcal{D}(\mathcal{M},\mathcal{N})$ on open affine subsets $U$ as
    \[\Gamma(U,\mathcal{D}(\mathcal{M},\mathcal{N}))=\Diff_{\mathcal{O}_X(U)}(\mathcal{M}(U),\mathcal{N}(U)),\]
    and localizing/gluing to obtain a sheaf (c.f. Hartshorne ex II.1.22 and II.5.?).
    This construction agrees with the above definition on the
    basis of open affine subsets of $X$, and hence, by Stacks Tag 009Q, yields the same sheaf.\footnote{We
    should perhaps move this trick into its own lemma, as it recieves use elsewhere. Or maybe not?}
\end{remark}

\begin{proposition}
    Order of derivations provides the $\mathcal{O}_X$-module $\mathcal{D}(\mathcal{M})$ with
    a filtration, making it a filtered sheaf of rings.
\end{proposition}
\begin{proof}
    Fix $U\subset X$ open. Sections of $\mathcal{D}(\mathcal{M})(U)$ are $\underline{\C}$-linear
    sheaf homomorphisms $\phi:\mathcal{M}|_U\to\mathcal{M}|_U$. Defining multiplication as composition
    of sections, we obtain a sheaf of rings. We give $\mathcal{D}(M)$ a filtration by specifying
    $F^k\mathcal{D}(\mathcal{M})$ to be the subsheaf consisting of morphisms which, over every affine
    open $V$, lie in $\Diff^k_{\mathcal{O}_X(V)}(\mathcal{M}(V),\mathcal{M}(V))$. It is easy to see
    that restriction maps preserve the filtration and that $\cup_kF^k\mathcal{D}(\mathcal{M})=\mathcal{D}(\mathcal{M})$.
    It remains to check that
    $F^i\mathcal{D}(\mathcal{M})\cdot F^j\mathcal{D}(\mathcal{M})\subset F^{i+j}\mathcal{D}(\mathcal{M})$.
    This follows from the filtration on $\Diff_R(M,M)\equiv\Diff_R(M)$ (for $M$ an $R$-module);
    if $\alpha\in\Diff^i_R(M),\beta\in\Diff^j_R(M),r\in R$,
    \[ [\alpha\beta,r] = \alpha\beta r-r\alpha\beta = \alpha[\beta,r] +[\alpha,r]\beta.  \]
    Now $[\alpha,r]$ is of order $\leqslant (i-1)$ and $[\beta,r]$ is of order $\leqslant(j-1)$, so induction
    yields the case of degree $i$ times an $R$-linear endomorphism $f$ (degree zero), but
    \[ [\alpha f, r]=\alpha fr-r\alpha f=\alpha rf-r\alpha f=[\alpha,r]f, \]
    so induction yields the case of the product two $R$-linear endomorphisms, which is another
    $R$-linear endomorphism, hence degree 0.
\end{proof}

\begin{remark}
    The sheaf of rings $\mathcal{D}(\mathcal{M})$ is not, in general, commutative or even
    almost-commutative (that is to say, $\gr_F \mathcal{D}(\mathcal{M})$ is not commutative).
    This is due to the fact that the endomorphism ring $\Diff_R^0(M)=\End_R(M)$ need not be
    commutative.\footnote{Perhaps it is worth mentioning the alternate definition of $\Diff$ here? We
    should learn more about the two definitions, although probably it is no big deal since
    they yield the same thing for $\mathcal{D}_X$.}
\end{remark}

\begin{proposition}
    If $\mathcal{M},\mathcal{N}$ are locally free, we obtain an isomorphism
    \[\mathcal{D}(\mathcal{M},\mathcal{N})=\mathcal{N}\otimes_{\mathcal{O}_X}\mathcal{D}_X\otimes_{\mathcal{O}_X}\mathcal{M}^*.\]
\end{proposition}
\begin{proof}
    Depending on the proof, this may or may not belong in the next section.
\end{proof}

\subsection{The sheaf $\mathcal{D}_X$}

Recall that we have defined $\mathcal{D}_X\equiv\mathcal{D}(\mathcal{O}_X,\mathcal{O}_X)$.
An often-used definition equivalent to the above is as follows.\footnote{why is this equivalent?} 

\begin{definition}
    Define $\mathcal{D}_X$ to be the sheaf associated to the subpresheaf of $\sHom_{\C_X}(\mathcal{O}_X,\mathcal{O}_X)$
    generated as a $\C_X$-algebra by (multiplication by) $\mathcal{O}_X$ and (derivation by)
    $\Theta_X$. Explicitly, for any open $U\subset X$, the sections of $\mathcal{D}_X(U)$ are
    the endomorphisms which are locally generated -- i.e. over an open cover $U_i$ of $U$ depending
    on the endomorphism -- by 
    $\{\tilde f,\tilde \partial\mid f\in\mathcal{O}_X(U_i),\partial\in\Theta_X(U_i)\}$
    subject to the relations
    \begin{enumerate}[(i)]
        \item $\tilde f_1+\tilde f_2=\widetilde{f_1+f_2}, \;\;\; \tilde f_1\tilde f_2=\widetilde{f_1f_2};$
        \item $\tilde \partial_1 +\tilde \partial_2=\widetilde{\partial_1+\partial_2}, \;\; [\tilde\partial_1,\tilde\partial_2]=\widetilde{[\partial_1,\partial_2]};$
        \item $\tilde f\tilde \partial=\widetilde{f\partial}, \;\;\; [\tilde\partial,\tilde f]=\widetilde{\partial(f)}$.
    \end{enumerate}
    \label{def:D_X2}
\end{definition}

\begin{proposition}
    Let $R$ be a commutative $\C$-algebra. Then the subalgebra of $\End_\C(R)$ generated
    by (multiplication by) $R$ and (derivation by) $\Der_\C(R)$ is isomorphic as a filtered
    $\C$-algebra to $\Diff_R(R)$.
\end{proposition}
\begin{proof}
    Let $\phi\in\Diff^k_R(R)$ be an operator of order $\leqslant k$ and define $\psi=\phi-\phi(1)$
    under the identification $\phi(1)\in R=\Diff^0_R(R)$, also of order $\leqslant k$.
    We show first, by induction, that $\psi$ satisfies the Leibniz rule.\footnote{base case?}
    In other words, $[\psi, r]$ is of order $\leqslant (k-1)$ for all $r\in R$, which by the
    induction hypothesis satisfies 
    \[ [\psi,r](ab)=[\psi,r](a)b+a[\psi,r](b).  \]
    Expanding this, we find that
    \begin{align*}
        [\psi,r](ab) &= \left( \psi(ra)-r\psi(a) \right)b+a\left( \psi(rb)-r\psi(b) \right)\\
        &= b\psi(ra) - rb\psi(a) + a\psi(rb) - ra\psi(b)
    \end{align*}
\end{proof}

This definition is especially useful due to the following coordinate description.

\begin{lemma}
    For each $p\in X$ there exists an affine open $U\ni p$ and sections
    $x_i\in\mathcal{O}_X(V),\partial_i\in\Theta_X(V)$ (for $1\leq i\leq n=\dim X$) satisfying
    \begin{enumerate}[(i)]
        \item $[\partial_i,\partial_j]=0$;
        \item $\partial_i(x_j)=\partial_{ij}$;
        \item $\Theta_X(V)=\bigoplus_{i=1}^n\mathcal{O}_X(V)\partial_i$.
    \end{enumerate}
\end{lemma}
\begin{proof}
    Since $X$ is smooth, the stalk $\mathcal{O}_{X,p}$ is a regular local ring. Then the maximal
    ideal $\fr m_p\subset \mathcal{O}_{X,p}$ is generated by $n$ elements, call them
    $x_1,\ldots,x_n\in\fr m_p$. Then $\Omega^1_{X,p}$ is a free $O_{X,p}$-module generated
    by $dx_1,\ldots, dx_n$.\footnote{explain this} Thus there exists an affine open neighboorhood
    $V$ of $p$ such that $\Omega_X(V)$ is a free $\mathcal{O}_X(V)$-module with basis
    $dx_1,\ldots,dx_n$.\footnote{why?} Fixing the dual basis $\partial_1,\ldots,\partial_n\in\Theta_X(V)$
    satisfying $dx_i(\partial_j)=\delta_{ij}$, we find that $\partial_j(x_i)=\delta_{ij}$.
\end{proof}

%define $\mathcal{D}_X$ to be the
%subsheaf of $\sHom_{\C_X}(\mathcal{O}_X,\mathcal{O}_X)$ generated as a $\C_X$-algebra by
%(multiplication by) $\mathcal{O}_X$ and (derivation by) $\Theta_X$. To check that this definition
%is equivalent to $\mathcal{D}(\mathcal{O}_X,\mathcal{O}_X)$, it suffices to check that the
%sections agree on the basis of affine opens. Explicitly, this requires checking that
%for $U\subset X$ affine open, $\Diff_{\mathcal{O}_X(U)}(\mathcal{O}_X(U))$ is isomorphic
%to the algebra generated by
%$\{\tilde f,\tilde \partial\mid f\in\mathcal{O}_X(U),\partial\in\Theta_X(U)\}$
%subject to the relations
%\begin{enumerate}[(i)]
%    \item $\tilde f_1+\tilde f_2=\widetilde{f_1+f_2}, \;\;\; \tilde f_1\tilde f_2=\widetilde{f_1f_2};$
%    \item $\tilde \partial_1 +\tilde \partial_2=\widetilde{\partial_1+\partial_2}, \;\; [\tilde\partial_1,\tilde\partial_2]=\widetilde{[\partial_1,\partial_2]};$
%    \item $\tilde f\tilde \partial=\widetilde{f\partial}, \;\;\; [\tilde\partial,\tilde f]=\widetilde{\partial(f)}$.
%\end{enumerate}
% 

%The following is an often-used, equivalent definition of $\mathcal{D}_X$, amenable to a
%coordinate description.
%\begin{definition}
%    We define the sheaf $\widetilde{\mathcal{D}_X}$ to be the sheaf of endomorphisms of $\mathcal{O}_X$
%    generated as a $\C_X$-algebra by (multiplication by) $\mathcal{O}_X$ and (derivation by) $\Theta_X$.
%    More explicitly, we define $\widetilde{\mathcal{D}_X}$ to be the subsheaf of $\sHom_{\C_X}(\mathcal{O}_X,\mathcal{O}_X)$
%    whose sections on an open set $U\subset X$ are morphisms whose evaluations at $U\cap V$, for
%    every open affine $V$, are in the algebra generated by $\{\tilde f,\tilde \partial\mid f\in\mathcal{O}_X(U\cap V),\partial\in\Theta_X(U\cap V)\}$
%    subject to the relations
%    \begin{enumerate}
%        \item[] $\tilde f_1+\tilde f_2=\widetilde{f_1+f_2}, \;\;\; \tilde f_1\tilde f_2=\widetilde{f_1f_2};$
%        \item[] $\tilde \partial_1 +\tilde \partial_2=\widetilde{\partial_1+\partial_2}, \;\; [\tilde\partial_1,\tilde\partial_2]=\widetilde{[\partial_1,\partial_2]};$
%        \item[] $\tilde f\tilde \partial=\widetilde{f\partial}, \;\;\; [\tilde\partial,\tilde f]=\widetilde{\partial(f)}$.
%    \end{enumerate}
%    Then $\widetilde{\mathcal{D}_X}$ is isomorphic to the sheaf $\mathcal{D}_X$ defined earlier.
%\end{definition}
%\begin{proof}
%    First we check that $\widetilde{\mathcal{D}_X}$ is indeed a sheaf. It is easy to see that
%    it is a separated subpresheaf (of rings) of the sheaf $\sHom_{\C_X}(\mathcal{O}_X,\mathcal{O}_X)$.
%\end{proof}

The following properties are fundamental.

\begin{proposition}\hspace{1mm}
    \begin{enumerate}[(i)]
        \item Each $F_i\mathcal{D}_X$ is a locally free $\mathcal{O}_X$-module of finite rank;
        \item $F_0\mathcal{D}_X=\mathcal{O}_X$, $F_i\mathcal{D}_X\cdot F_j\mathcal{D}_X=F_{i+j}\mathcal{D}_X$;
        \item If $P\in F_i\mathcal{D}_X$ and $Q\in F_j\mathcal{D}_X$, then $[P,Q]\in F_{i+j-1}\mathcal{D}_X$;
        \item $\mathcal{D}_X$ is almost-commutative, i.e. $\gr_F\mathcal{D}_X$ is commutative.
    \end{enumerate}
\end{proposition}
\begin{proof}
    
\end{proof}

\begin{remark}
    This filtration on $\mathcal{D}_X$ (inherited from the filtration on $\mathcal{D}(\mathcal{M})$)
    is often called the \textbf{geometric filtration}.\footnote{what is the arithmetic filtration?}
\end{remark}

\begin{example}
    Provide some examples of $\mathcal{D}_X$-modules.
\end{example}


\subsection{The flag variety of $G$}

\begin{definition}
    Let $V$ be an $n$-dimensional complex vector space. We define the \textbf{Grassmanian} $\Gr(d,V)$
    to be the set of $d$-dimensional subspaces of $V$. The following result shows that $\Gr(d,V)$ is,
    in fact, a projective variety.
\end{definition}

\begin{proposition}
    The set $\Gr(d,V)$ has the structure of a projective variety.
\end{proposition}
\begin{proof}
    Define the map $f:\Gr(d,V)\to\PP(\bigwedge^dV)$ as taking any $d$-dimensional subspace $W\subset V$ to the
    line $\bigwedge^d W\subset \bigwedge^dV$. More explicitly, if we fix a basis $v_1,\ldots, v_d$ for $W$,
    we find that $f(W)=[v_1\wedge\cdots\wedge v_d]$. Now it suffices to show that $f$ is injective with closed image.

    Injectivity is straightforward: suppose $W,W'\subset V$ are a pair of $d$-dimensional subspaces of $V$
    with bases $v_1,\ldots,v_d$ and $v_1',\ldots,v_d'$ respectively. If $f(W)=f(W')$, we must have that
    $v_1\wedge\cdots\wedge v_d=\lambda v_1'\wedge\cdots\wedge v_d'$ for some $\lambda\in\C$. This implies
    that $W=W'$, since $W$ (resp. $W'$) is exactly the set of vectors $w\in V$ (resp. $w'\in V$) satisfying
    $w\wedge v_1\wedge\cdots\wedge v_d=0$ (resp. $w'\wedge v_1'\wedge\cdots\wedge v_d'=0$).

    To show that the image of $f$ is closed, we first remark that the image of $f$ is (up to scaling) the subset
    of totally decomposable vectors in $\bigwedge^dV$: wedge products $v_1\wedge\cdots\wedge v_d$
    of $d$ linear factors $v_i\in V$. It is easy to see that $\omega\in\bigwedge^dV$ is totally decomposable if
    and only if the space of vectors $v$ dividing it, i.e. $v\wedge\omega=0$, is $d$-dimensional.
    Hence $[\omega]$ is in the image of $f$ if and only if the map $\cdot\wedge\omega:V\to \bigwedge^{d+1}V$
    taking $v\mapsto v\wedge \omega$ is of rank $n-d$. But since $\omega\in\bigwedge^dV$ can never
    be divisble by more than $d$ linearly independent vectors of $V$, the rank of $\cdot\wedge\omega$
    is always $\geqslant n-d$. Thus
    \[ [\omega]\in\im f\text{ if and only if }\rk(\cdot\wedge\omega)\leqslant n-d.\]
    Thus $\Gr(d,V)=\im f$ is determined by the vanishing of the determinants of $(n-d+1)\times(n-d+1)$ minors
    of the matrix $\cdot\wedge\omega\in\Hom(V,\bigwedge^{d+1}V)$.
    Since the map $\omega\mapsto \cdot\wedge\omega$ is linear, the matrix elements of $\cdot\wedge\omega$
    are linear in coordinates of $\bigwedge^dV$, and hence the determinants are homogeneous, degree $(n-d+1)$
    polynomials in the coordinates of $\bigwedge^dV$. This shows that $\Gr(d,V)$ is indeed
    Zariski-closed in $\PP(\bigwedge^dV)$.
\end{proof}

\begin{definition}
    Let $V$ be an $n$-dimensional complex vector space. We define a $k$-\textbf{flag}
    in $V$ to be a sequence of proper inclusions of subspaces
    \[\{0\}=V_0 \subset V_1 \subset \cdots \subset V_k=V.\]
    We say that $d_I=(\dim V_1,\ldots,\dim V_k)$ is the \textbf{signature} of the flag.
    We define the \textbf{flag variety} $\Fl(d_I,V)$ to be the set of all $k$-flags in $V$ with signature $d_I$.
\end{definition}

\begin{remark}
    We will typically be working with \textbf{complete flag varieties} $\Fl(V)=\Fl( (1,\ldots,n), V)$.
\end{remark}

\begin{proposition}
    The set $\Fl(d_I,V)$ has the structure of a projective variety.\footnote{Reference: http://sierra.nmsu.edu/morandi/notes/grassmannian.pdf}
\end{proposition}
\begin{proof}
    We can write
    \[\Fl(d_I,V)=\{(V_1, \cdots,V_k)\in\Gr(d_1,V)\times\cdots\times\Gr(d_k,V) \mid V_{i}\subset V_{i+1}\}.\]
    To prove that $\Fl(d_I,V)$ is a subvariety of the product of Grassmannians,
    it suffices to show that $\{(W_1,W_2)\in\Gr(d_1,V)\times\Gr(d_2,V) \mid W_1\subset W_2\}$ is closed
    in $\Gr(d_1,V)\times\Gr(d_2,V)$ where $d_1<d_2$. The set $\Fl(d_I,V)$ is then simply the intersection
    of such sets.
    
    Let $v_1,\ldots, v_{d_1}$ be a basis for $W_1$
    and $v_1',\ldots, v_{d_2}'$ be a basis for $W_2$ and write $\omega_1=v_1\wedge\cdots\wedge v_{d_1}$
    and $\omega_2=v_1'\wedge\cdots\wedge v_{d_2}'$. As in the proof above, we consider the map
    $\phi:\cdot\wedge\omega_1\oplus\cdot\wedge\omega_2:V\to \bigwedge^{d_1+1}V\oplus\bigwedge^{d_2+1}V$.
    By definition, $\ker(\cdot\wedge\omega_1)=W_1$ and $\ker(\cdot\wedge\omega_2)=W_2$, so
    $\ker\phi=W_1\cap W_2$. The rank-nullity theorem now yields
    \begin{align*}
        \rk\phi &= \dim V - \dim\left( \ker\phi \right)\\
        &= \dim V - \dim(W_1\cap W_2)\\
        &\geqslant \dim V - \dim W_1.
    \end{align*}
    Furthermore, we have $W_1\subset W_2$ if and only if $W_1\cap W_2=W_1$, from which we conclude that
    $W_1\subset W_2$ if and only if $\rk\phi\leqslant\dim V-\dim W_1$. Now the argument
    from the previous proof -- that the determinant of certain minors of the matrix of $\phi$ yields
    homogeneous polynomials whose vanishing characterizes the rank condition -- completes
    the proof.
\end{proof}

\begin{example}
    Consider the action of $\SL_2$ on $\PP^1$, with homogeneous coordinates $[z_1:z_2]$,
    given by $g[z_1:z_2]=[g_{11}z_1+g_{12}z_2:g_{21}z_1+g_{22}z_2]$. It is easy to check
    that the stabilizer of the point $[1:0]$ is the set of all upper diagonal matrices in
    $\SL_2$. This is precisely the Borel subgroup $B$ of $\SL_2$.\footnote{make sure the
    Borel subgroup has been introduced, and explain this claim.} Moreover, the orbit of $[1:0]$
    is clearly the whole variety $\PP^1$. Hence we obtain a bijection between the $\SL_2/B$ of
    cosets and $\PP^1$, which endows $\SL_2/B$ the structure of the complete flag variety
    $\Fl(\C^2)\cong\PP^1$.

    More generally, consider the natural action of $\SL_n$ on the complete flag variety
    $\Fl(\C^n)$ as
    \[g(V_1,\ldots, V_n) = (gV_1,\ldots, gV_n)\]
    for any complete flag $(V_1,\ldots,V_n)$. This action is transitive. Let $(W_1,\ldots,W_n)$ be
    any other complete flag and $\{v_1,\ldots,v_n\}\subset \C^n$ such that for each $i\leqslant n$
    the first $i$ vectors form a basis for $V_i$, and $\{w_1,\ldots,w_n\}\subset \C^n$ such that
    for each $i\leqslant n$ the first $i$ vectors form a basis for $W_i$. Then there exists
    a $\tilde g\in\GL_n$ such that $\tilde gv_i=w_i$. Dividing $g=\tilde g/\det \tilde g$,
    we find $g\in\SL_n$ taking $(V_1,\ldots,V_n)$ to $(W_1,\ldots,W_n)$. Now consider the
    complete flag determined by the basis vectors $\{v_1,\ldots,v_n\}$ of $\C^n$. A bit of algebra
    shows that the stabilizer of this flag is the Borel subgroup $B$ of upper diagonal
    matrices. Hence the coset space $\SL_n/B$ has the structure of the complete flag variety
    $\Fl(\C^n)$.
\end{example}

\begin{remark}
    Even more generally, one obtains not-necessarily complete, partial flag varieties
    from quotients of the form $\SL_n/P$ where $P$ is the subgroup of block-diagonal
    matrices with block sizes corresponding to the signature of the flag variety.\footnote{Is
    this true? Also: perhaps we should mention something about roots and choice of simple
    roots in these examples? Of course, the results are independent.} These subgroups
    are examples of parabolic subgroups.\footnote{reference Borel}
\end{remark}

\subsection{Equivariant sheaves}
\subsection{Borel-Weil-Bott}

%\printbibliography

\end{document}
