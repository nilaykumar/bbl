\chapter{Preliminaries}

\blurb{We begin with a review of derivations and differential operators in the local case. 
    We then introduce the sheaf $\mathscr{D}_X$ and study basic properties of $\mathscr{D}_X$-modules.}

\section{Local constructions}

\begin{definition}
    We say that a commutative ring $R$ is \textbf{affine} if it is
    the coordinate ring of an affine complex algebraic variety, i.e.
    if $R$ is an integral domain of finite type over $\C$.
\end{definition}

\begin{definition}
    Let $R$ be affine, and $M,N$ two $R$-modules.\sidenote{The affine condition is not strictly necessary
    for many of our definitions,
    but it will be useful when working over complex algebraic varieties.}
    We define the module of
    differential operators $\Diff_R(M,N)$ from $M$ to $N$ to be the homomorphisms $d\in\Hom_\C(M,N)$
    such that there exists an $k\in\N$ with any set of $k+1$ elements of $R$ satisfying
    \begin{equation}
        (\ad r_0)(\ad r_1)\cdots(\ad r_k)(d)=0,
        \label{eq:ad}
    \end{equation}
    where $(\ad r)(d)=[r,d]=rd-dr$. In this case, we say that $d$ is of order $\leqslant k$.
    Note that the submodule of differential operators of order $\leqslant 0$, in particular,
    is simply $\Hom_R(M,N)$. We will write $D(R)$ for the module $\Diff_R(R,R)\subset\End_{\C}(R,R)$.
    \label{def:derivations}
\end{definition}

\begin{remark}
    The order of a differential operator provides a natural filtration of the module $\Diff_R(M,N)$.
    This yields the following inductive definition of $\Diff_R(M,N)$, equivalent to the one
    above\sidenote{In this formulation, these are typically known as Grothendieck differential
    operators.}: % TODO should we mention the other definition using im(R->Hom(M,N))?
    \begin{enumerate}
        \item[] $\Diff^{i}_R(M,N) = 0; \;\;\;(i<0)$
        \item[] $\Diff^k_R(M,N) = \{u\in\Hom_\C(M,N)\mid (\ad r)(u)\in \Diff^{k-1}_R(M,N)\};$
        \item[] $\Diff_R(M,N)=\bigcup_k\Diff^k_R(M,N)$.
    \end{enumerate}
\end{remark}

\begin{lemma}
    The module $D(R)=\Diff_R(R,R)$ forms a filtered ring under composition of
    differential operators.
    \label{lemma:filtration}
\end{lemma}
\begin{proof}
    That $D(R)$ forms a ring is clear; it suffices to show that $D^i(R)\cdot D^j(R)\subset D^{i+j}(R)$.
    If we have $\alpha\in D^i(R),\beta\in D^j(R),r\in R$, then
    \[ [\alpha\beta,r] = \alpha\beta r-r\alpha\beta = \alpha[\beta,r] +[\alpha,r]\beta.  \]
    Now $[\alpha,r]$ is of order $\leqslant (i-1)$ and $[\beta,r]$ is of order $\leqslant(j-1)$, so induction
    yields the case of degree $i$ times an $R$-linear endomorphism $f$ (of degree zero). Now
    \[ [\alpha f, r]=\alpha fr-r\alpha f=\alpha rf-r\alpha f=[\alpha,r]f, \]
    so another induction yields the case of the product two $R$-linear endomorphisms, which is another
    $R$-linear endomorphism, hence degree $0$.
\end{proof}

\begin{definition}
    Let $M$ be an $R$-module. Recall that $\Der_\C(R,M)$ is the (left) $R$-module of $\C$-module homomorphisms
    $\delta:R\to M$ satisfying the product rule $\delta(rs)=r\delta(s)+\delta(r)s.$
    The \textbf{derivation ring} $\Delta(R)$ of $R$ is the subring of $\End_\C(R,R)$ generated
    by $R$ and $\Der_\C(R,R)$.
\end{definition}

\begin{lemma}
    The derivation ring $\Delta(R)$, under the filtration where $\Delta^i(R)$ is the
    $R$-submodule generated by all products of at most $i$ derivations, is a filtered
    subring of the ring of differential operators $D(R)$.
    \label{lemma:filtsubring}
\end{lemma}
\begin{proof}
    It is clear that $\Delta^i(R)\cdot\Delta^j(R)\subset\Delta^{i+j}(R)$,\sidenote{In fact,
    we have equality.} so the filtration on $\Delta(R)$ is well-defined. It remains to show
    that $[\Delta^i(R),r]\subset\Delta^{i-1}(R)$. We proceed inductively: the result clearly
    holds for $i=0$. Now suppose it holds for some $i-1$ and let $d\in\Delta^{i}(R)$. Assume,
    without loss of generality, that we can write $d=d_1\cdots d_i$ with each
    $d_j\in\Der_\C(R)$.\sidenote{In general, $\Delta^i(R)$ is a sum of terms
    each with a product of a different number $\leqslant i$ of derivations.
    We can drop all terms with fewer than $i$ derivations by the induction step.}
    Then, for $r,s\in R$, we find that
    \begin{align*}
        [d_1\cdots d_i,r]s &= d_1\cdots d_irs-rd_1\cdots d_is\\
        &= d_1\cdots d_{i-1}rd_is+d_1\cdots d_{i-1}[d_i,r]s-rd_1\cdots d_is\\
        &= \left([d_1\cdots d_{i-1},r]d_i+d_1\cdots d_{i-1}[d_i,r]\right)s.
    \end{align*}
    The term in parentheses, by the induction step, lives in $\Delta^{i-1}(R)$,
    and hence $[\Delta^i(R),r]\subset\Delta^{i-1}(R)$, as desired.
\end{proof}

\begin{example}[Weyl algebra]
    % TODO write this
    Define the Weyl algebra and show that in this case the two algebras are
    isomorphic.
\end{example}

\begin{lemma}
    The inclusion $\Delta(R)\subset D(R)$ induces an isomorphism of
    $R$-modules $\Delta^1(R)\cong D^1(R)$.
    \label{label:firstorder}
\end{lemma}
\begin{proof}
    By definition, $\Delta^1(R)\cong R\oplus\Der_\C R\subset D^1(R)$. Take any $\alpha\in D^1(R)$
    and let $\beta=\alpha-\alpha(1)$. By definition, $[ [\beta,r],s]=0$ for any $r,s\in R$.
    Evaluating both sides of this equation on 1, we find
    \[0 = \beta(rs)-r\beta(s)-s\beta(r),\]
    i.e. that $\beta$ acts as a derivation. Thus $\alpha=\alpha(1)+\beta\in\Delta^1(R)$.
\end{proof}

\begin{example}
    In general, the two algebras are not isomorphic.
\end{example}

\begin{theorem}
    If $R$ is regular, then the inclusion $\Delta(R)\subset D(R)$ is an isomorphism.
    \label{label:derivationiso}
\end{theorem}

\begin{example}
    Show that these are isomorphic for affine space.
\end{example}

Why is this theorem useful? Because we can use the coordinate description of differential
operators in the global sheafified case.

\begin{lemma}
    Let $L$ be an extension field of $\C$ and $K$ be an algebraic extension of $L$. If
    $N$ is a $K$-module then any $f\in\Diff_L(L,N)$ has at most one extension to a
    differential operator in $\Diff_K(K,N)$.
    \label{lemma:hart1}
\end{lemma}
\begin{proof}
\end{proof}

\begin{corollary}
    Suppose that $L$ is an extension field of $\C$ and suppose that $K$ is an algebraic
    extension of $L$. Suppose that $f:K\to K$ is a differential operator such that $fw-wf$
    has order $\leqslant p$ for all $w$ in $L$. Then $f$ has order $\leqslant p+1$.
\end{corollary}
\begin{proof}
\end{proof}

\begin{lemma}
    Let $R$ be an affine ring with $S\subset R$ a multiplicatively closed subset
    (not containing zero) and $P\in D^i(R)$ is a differential operator or order $\leqslant i$.
    Denoting the commutator $[P,r]$ by $\ad'(r)(P)$, we claim that if $a,b\in R$
    and $s,t\in S$ such that $at=bs$, then 
    \begin{equation}
        \sum_{j=0}^i(-1)^js^{-j-1}(\ad's)^j(P)(a) = \sum_{j=0}^i (-1)^jt^{-j-1}(\ad't)^j(P)(b)
        \label{eq:loc_diff}
    \end{equation}
    in the localization (under the canonical injection $R\subset S^{-1}R$).
    \label{hart:lemma2}
\end{lemma}
\begin{proof}
    We induct on the order $i$ of $P$. For $i=0$, proving the above identity amounts to
    showing that $s^{-1}P(a)=t^{-1}P(b)$, which follows immediately from the $R$-linearity
    of zeroth-order differential operators. So suppose that $P$ is of order $i\geqslant 1$.
    Invoking the induction hypothesis on $(\ad's)(P)$, we have
    \begin{align}
        \sum_{j=0}^{i-1} (-1)^js^{-j-1}(\ad's)^{j+1}(P)(a) &= \sum_{j=0}^{i-1} (-1)^jt^{-j-1}(\ad't)^j[P,s](b).
        \label{eq:hart_star}
    \end{align}
    Applying the Jacobi identity repeatedly on each term of the right hand side and using the
    fact that $[s,t]=0$, we can commute the $[\cdot, s]$ to obtain
    \begin{align*}
        \text{RHS (\ref{eq:hart_star})} &= \sum_{j=0}^{i-1} (-1)^jt^{-j-1}(\ad's)(\ad't)^j(P)(b)\\
        &= t^{-1}P(sb)-t^{-1}sP(b)-t^{-2}[P,t](sb)+t^{-2}s[P,t](b)+\\
        &\indent+ t^{-3}[ [P,t],t](sb) - t^{-3}s[ [P,t],t](b) + \cdots\\
        &= -s\cdot \text{RHS (\ref{eq:loc_diff})}+t^{-1}P(at)-t^{-2}[P,t](at)+\\
        &\indent+t^{-3}[ [P,t],t](at)-\cdots 
    \end{align*}
    We now claim that
    \[P(a)=t^{-1}P(at)-t^{-2}[P,t](at)+t^{-3}[ [P,t],t](at)-\cdots,\]
    i.e. the sum of all but the first terms in the last expression above for RHS (\ref{eq:hart_star})
    is $P(a)$. This is proved by induction: if $i=0$ then the equation holds because $t^{-1}P(at)=P(a)$
    by $R$-linearity of $P$ and all higher commutators vanish. Invoking the induction
    hypothesis on $[P,t]$ yields the equality
    \[ [P,t](a)=t^{-1}[P,t](at)-t^{-2}[ [P,t],t](at)+\cdots.\]
    Expanding the commutator on the left and multiplying both sides by $t^{-1}$ yields
    the desired result. This, together with the results above yields Eq. (\ref{eq:loc_diff}).
\end{proof}

\begin{remark}
    Explain intuitively this formula for the action of differential operators on the
    localization.
\end{remark}


% TODO make the treatment of derivations more streamlined
%
%\begin{definition}
%    Let $F$ be the free $R$-module on symbols $dr$ for $r\in R$ and let $N$
%    be the submodule of $F$ generated by $d\lambda,d(r+s)-dr-ds,$ and $d(rs)-r(ds)-s(dr)$
%    for $\lambda\in\C$ and $r,s\in R$.
%    We denote by $\Omega=\Omega_\C(R)=F/N$ the module of \textbf{K\"ahler differentials}
%    of $R$, and by $d=d_R:R\to R\to\Omega_\C(R)$ taking $a\mapsto da$ the \textbf{universal derivation}
%    of $R$.\sidenote{Reference Matsumura or Hartshorne for derivations.} % TODO reference this
%\end{definition}
%
%\begin{lemma}\hfill
%    \begin{enumerate}[(i)]
%        \item Let $M$ be an $R$-module and $\delta\in\Der_\C(R,M)$ a derivation.
%            Then there is a unique $\phi\in\Hom_R(\Omega_\C(R),M)$ such that $\delta=\phi d$.
%        \item The map $\Hom_R(\Omega,M)\to\Der_\C(R,M)$ given by $\phi\mapsto \phi d$ is
%            an $R$-module isomorphism. Thus $\Der_\C R\cong\Hom_R(\Omega,R)=\Omega^\vee$.
%    \end{enumerate}
%    \label{lemma:universal}
%\end{lemma}
%\begin{proof}
%    % TODO finish this proof
%\end{proof}
%


