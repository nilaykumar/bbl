\chapter{Linear Algebraic Groups}

\blurb{Blurb here.}

\begin{definition}
    An \textbf{algebraic group} is an algebraic variety $G$ together with an identity
    element $e\in G$, multiplication $\mu:G\times G\to G$, and inversion
    $i:G\to G$, such that $\mu$ and $i$ are maps of algebraic varieties
    with respect to which $G$ forms a group. A morphism of algebraic groups
    is a morphism of algebraic varieties that is simultaneously a group
    homomorphism.\sidenote{For affine algebraic groups, one can reformulate this
    definition in terms of the coordinate ring $\mathcal{O}_G(G)$. Reference
    Borel or Mukai.}
\end{definition}

\begin{example}
    \hspace{1mm} 
    \begin{enumerate}[(i)]
        \item The affine space $\C$ has a natural additive group structure --
            we denote the resulting algebraic group by $\G_a$.
        \item We denote by $\GL_n$ the general linear group of all invertible
            $n\times n$ matrices. $\GL_n$ has the structure of an open subvariety
            of $\C^{2n}$: simply take $\GL_n=\Specm\C[x_{11},\ldots,x_{nn},D^{-1}]$
            where $D=\det(x_{ij})$. The special case $\GL_1=\C^\times$ is sometimes
            called the multiplicative group $\G_m$.
    \end{enumerate}
\end{example}

\begin{theorem}
    Relation between affinity and linearity of algebraic groups.
\end{theorem}

\begin{theorem}
    Smoothness and other properties
\end{theorem}

\begin{definition}
    Solvability, etc.
\end{definition}

\section{Semisimple groups}
\section{Bruhat decomposition}
\section{Representations}
