\chapter{Linear Algebraic Groups}

\blurb{Blurb here.}

\begin{definition}
    An \textbf{algebraic group} is an algebraic variety $G$ together with an identity
    element $e\in G$, multiplication $\mu:G\times G\to G$, and inversion
    $i:G\to G$, such that $\mu$ and $i$ are maps of algebraic varieties
    with respect to which $G$ forms a group. A morphism of algebraic groups
    is a morphism of algebraic varieties that is simultaneously a group
    homomorphism.
\end{definition}

\begin{example}
    Provide some examples.
\end{example}

\begin{theorem}
    Relation between affinity and linearity of algebraic groups.
\end{theorem}

\begin{theorem}
    Smoothness and other properties
\end{theorem}

\section{Semisimplicity}
\section{Root systems}
\section{Representations}
